\section*{Введение}

%Автоматизированные маркет-мейкеры (AMM) представляют собой алгоритмические механизмы ценообразования, получившие широкое распространение в децентрализованных финансах для организации обмена криптоактивами без традиционной книги заявок.
Автоматизированные маркет-мейкеры (AMM) представляют собой структуры для обмена активами с заданным алгоритмами ценообразования, получившие широкое распространение в децентрализованных финансах в роли механизма для обмена криптоактивами без традиционной книги заявок.
%Классический пример — пул ликвидности с постоянным произведением резервов, предложенный для бирж типа Uniswap, где цена актива определяется по формуле $x \cdot y = const$ \cite{mohan2022defi} при объемах двух токенов $x$ и $y$ в пуле.
Существует множество различных алгоритмов ценообразования. Например с константной суммой объёмов двух активов, или константным произведением объёмов двух активов (Uniswap) \cite{mohan2022defi}. Последний получил широкое распространение и будет исследован и в этой статье. 
%В традиционных финансовых рынках роль маркет-мейкеров обычно выполняют специализированные участники, котирующие цены покупки/продажи и зарабатывающие на спреде.
%!
На ряду с такими простыми алгоритмами существуют и более сложные. К примеру, HFMM \cite{egorov2021cryptoswap}. Он сочетает в себе преимущества предыдущих двух алгоритмов и так же будет исследован далее. 
В традиционных же финансовых рынках роль маркет-мейкеров обычно выполняют специализированные участники, котирующие цены покупки/продажи и зарабатывающие на спреде.

Возникает ряд вопросов. К примеру, способны ли алгоритмы AMM обеспечить аналогичную или лучшую устойчивость рынка по сравнению с классическим маркет-мейкером, особенно в шоковых ситуациях. Также возникает вопрос о качестве ликвидности при использовании AMM.

Актуальность исследования подтверждается растущим интересом к применению идей DeFi в традиционных финансах.
Так, в проекте BIS Project Mariana (2023) \cite{bis2023mariana} была успешно протестирована возможность использования AMM для автоматизации торгов и расчетов на межбанковском валютном рынке с цифровыми валютами центральных банков.
%!
В статье был использован алгоритм HFMM.
Отчет Mariana отмечает, что AMM может повысить эффективность рынка и снизить расчетные риски, хотя и требует предварительного депонирования ликвидности.
На фоне этих результатов важно понять, как внедрение AMM влияет на поведение рынка и его устойчивость к шокам.
Настоящая работа посвящена исследованию этой проблемы с помощью агентно-ориентированной модели рынка ценных бумаг.

Цель исследования -- сравнить влияние автоматизированного маркет-мейкера и классического маркет-мейкера на устойчивость рынка в условиях внешних шоков.
Для достижения этой цели сформулированы следующие гипотезы:

%---%

\subsection*{Первая гипотеза}

Использование AMM повышает устойчивость рынка к резким ценовым шокам по сравнению с классическим маркет-мейкером.
Выражается это в более быстром восстановлении ценового равновесия после шока и меньшем отклонении новой равновесной цены от исходного уровня.

\subsection*{Вторая гипотеза}

Использование AMM повышает устойчивость рынка при массовом уходе участников (шоке ликвидности) по сравнению с классическим маркет-мейкером.
В этой гипотезе также предполагается, что рынок с AMM быстрее восстанавливается после сокращения числа активных трейдеров, а равновесная цена меняется менее существенно.

\subsection*{Третья гипотеза}
В стационарном режиме (без шоков) AMM обеспечивает более узкий и стабильный спред и большую глубину книги, чем классический маркет-мейкер.
Для этого сравниваются средний и относительный спред ($ask-bid$ и $\frac{ask-bid}{mid}$), их стандартные отклонения, средний объём на лучших котировках, глубина в узком ценовом коридоре вокруг мида ($\pm 1\%$), среднее число заявок в книге (bid/ask) и отношение спреда к суммарному объёму на лучших котировках.
Такая постановка позволяет оценить, выигрывает ли AMM по качеству предоставляемой ликвидности даже без воздействия шоков.\\

Далее в статье описаны методология и модель исследования \cite{ai360project2025}, проведены численные эксперименты для проверки выдвинутых гипотез, сравниваются результаты для AMM и классического маркет-мейкера, а также представлены выводы об устойчивости рынка.

%TODO: IT; DONE; JUST COMMENT
%(ПЕРЕДЕЛАТЬ ЭТОТ ПОДРАЗДЕЛ) Отдельно рассматривается гипотеза о базовом качестве ликвидности без внешних шоков: в стационарном режиме AMM обеспечивает более узкий и стабильный спред и большую глубину книги, чем классический маркет-мейкер. Для её проверки используются симуляции без шоковых событий (специальный конфигурационный файл), где сравниваются агрегаты по всему прогону: средний и относительный спред (ask--bid и (ask--bid)/mid), их стандартные отклонения, средний объём на лучших котировках, глубина в узком ценовом коридоре вокруг мида ($\pm 1\%$) и среднее число заявок в книге (bid/ask). Статистическая значимость проверяется парными тестами на равенство средних (t-тест или Wilcoxon при ненормальности) и тестом Левена на равенство дисперсий. Метрика ``спред на объём'' рассчитывается как отношение спреда к суммарному объёму на лучших котировках и отражает, насколько дорогой становится предельная ликвидность. Такая постановка позволяет оценить, выигрывает ли AMM по качеству предоставляемой ликвидности даже без воздействия шоков.

\section*{Методология исследования}

Исследование проведено с использованием моделирования на основе агентно-ориентированной модели (ABM) финансового рынка.
В модели имитируется торговля одним активом на бирже с книгой заявок.
Реализовано несколько типов агентов-трейдеров, моделирующих различных участников рынка:
\begin{itemize}
    \item Random -- случайные трейдеры, подающие произвольные заявки на покупку или продажу.
    \item Fundamentalist -- фундаментальные трейдеры, оценивающие справедливую цену акции на основе ожидаемых дивидендов и совершающие сделки при значительном отклонении цены от этой оценки.
    \item Chartist -- трейдеры-техники, ориентирующиеся на ценовые тренды (покупают при растущем тренде – оптимисты, или продают при падающем – пессимисты).
    \item Universalist -- универсальные трейдеры, способные переключаться между фундаментальной и технической стратегиями в зависимости от рыночной ситуации.
    \item MarketMaker -- классический маркет-мейкер, поддерживающий двусторонние котировки: размещает лимитные заявки на покупку и продажу с определенным спредом, получая прибыль за счет разницы цен.
    Такой агент обеспечивает ликвидность, но может уходить с рынка или расширять спред при сильной волатильности.
\end{itemize}

%%ПЕРЕПИСАТЬ
Для моделирования рынка взята за основу существующая ABM-реализация одномерного фондового рынка \cite{1d-abm2025}, расширенная новыми типами AMM-агента.
AMM-агент действует по той же логике, что и пул ликвидности с заданным алгоритмом ценообразования (Uniswap или HFMM).
Важно подчеркнуть, что реализуется именно математическая модель AMM, без привязки к блокчейн-инфраструктуре или иным аттрибутам мира блокчейна.

Изначально AMM-агенту необходимо задать объемы двух условных активов ($A$ и $B$) в пуле ликвидности. В симуляторе это количество ценных бумаг и их суммарная стоимость.
Пусть к началу симуляции у агента в пуле $x$ единиц актива $A$ и $y$ единиц актива $B$ – эти величины выбраны пропорционально начальной рыночной цене актива $A$ (т.е. $\frac{y}{x}$ соответствует стартовой цене $P_{AB}$).

\subsection*{Механизм ценообразования AMM в Uniswap}
Обозначим текущие запасы актива $A$ как $x$, актива $B$ как $y$.
При продаже некоторого количества $t$ актива $A$ в обмен на актив $B$ из пула (т.е. трейдер покупает $t$ единиц $A$ у AMM и платит ему $k$ единиц $B$), новые запасы составят $x' = x-t$ и $y' = y+k$.
Алгоритм поддерживает инвариант $x' \cdot y' = x \cdot y$.
Из этого условия находится величина $k$, необходимая для покупки $t$ актива $A$:

$$(x - t)(y + k) = xy$$

откуда

$$k = \frac{xy}{x - t} - y \qquad (1)$$

Формула (1) задает объем актива $B$ -- $k$, который должен заплатить покупатель за объём $t$ актива $A$.
Из этой же зависимости выводится и правило обновления цены.
До сделки цена актива $A$ в единицах $B$ равнялась $P_{AB} = \frac{y}{x}$.
После того, как из пула изъято $t$ актива $A$ и добавлено $k$ актива $B$, новая цена актива $A$, т.е. $P'_{AB}$ (исчисляяемая в объёме актива $B$) определяется отношением запасов $P'_{AB} = \frac{y+k}{x-t}$.
Подставив выражение (1) для $k$, получим:

$$P'_{AB} = \frac{xy}{(x - t)^2} \qquad (2)$$

Аналогично, цена актива $B$ в пересчете на $A$ (т.е. $P'_{BA}$ в обозначениях) падает с $\frac{x}{y}$ до $\frac{(x-t)^2}{xy}$.
Формулы симметричны относительно обмена активами: покупка актива $A$ за $B$ эквивалентна продаже актива $B$ за $A$.
Отрицательное $t$ в формуле (1) будет означать обратную ситуацию – продажу актива $A$ в пул (покупку актива $B$ у AMM).

Таким образом, AMM-агент всегда готов выполнить сделку по динамически изменяющейся цене согласно (2).
В отличие от классического маркет-мейкера, чьи заявки и спред фиксированы или изменяются дискретно, AMM предоставляет непрерывную кривую цен в зависимости от размера сделки.
Однако для имитации работы AMM в среде с книгой заявок агент должен выставлять лимитные ордера, приблизительно повторяя форму этой кривой.

\subsection*{Реализация AMM-стратегии через лимитные и рыночные ордера в Uniswap}

AMM-агент анализирует текущую книгу заявок биржи и совершает рыночные сделки, если рыночная цена отклонилась от цены пула.
Алгоритм действий агента при текущей цене актива в структуре $P_{AB}$ выглядит так:

\begin{itemize}
    \item Если в книге имеются активные заявки на покупку $A$ по цене выше текущего $P_{AB}$ (оно обновляется после каждой сделки), агент продаёт актив $A$ по этой заявке, повышая $x$ в формуле (1) и тем самым увеличивая цену в пуле до уровня цен заявок, пытаясь сделать эти цены равными.
    Заявки перебираются в порядке неубывания цены.
    При равенстве цен порядок заявок не важен.
    Каждая заявка заполняется полностью или частично.
    Объём продажи $t$ для заполнения очередной заявки рассчитывается из условия достижения ценой пула уровня цены заявки $P^{(order)}_{AB}$: $t = x − \sqrt{P_{AB} \cdot P^{(order)}_{AB}}$ (решение уравнения $P'_{AB} = P^{(order)}_{AB}$ по $t$).
    Если рассчитанный $t$ превышает объём $P_{order}$ заявки, заявка исполняется полностью (AMM продаёт соответствующее количество $A$), иначе заявка исполняется частично на объём $t$, после чего текущая цена пула сравнивается со следующей заявкой.
    \item Если в книге есть заявки на продажу A по цене ниже текущего $P$, агент поступает аналогично с другой стороны: покупает актив $A$, уменьшая $x$ и тем самым понижая цену пула до уровня встречных заявок.
    Объём покупки рассчитывается по формуле $\sqrt{P_{AB} \cdot P^{(order)}_{AB}} - x$ (т.е. просто то же $t$ из предыдущего пункта с противоположным знаком), исходя из требуемого снижения цены до $P_{order}$.
    Далее заявки на продажу $P_{order}$ перебираются от самых низких цен, заполняясь полностью либо на рассчитанный объём аналогично предыдущему пункту.
\end{itemize}

Помимо рыночных операций, AMM-агент размещает лимитные ордера для непрерывного присутствия в книге заявок.
Суть стратегии – поддерживать на рынке ликвидность, распределенную вдоль кривой цен (формулы типа (2)).
В простейшем варианте агент выставляет серию мелких ордеров на покупку и продажу, расходуя свой запас актива $A$ равномерно по диапазону цен выше текущего уровня, а запас $B$ – по диапазону цен ниже текущего уровня.
Таким образом, даже если в текущий момент нет подходящих рыночных заявок для немедленной сделки, AMM-агент все равно обеспечивает наличие ордеров в книге, как это делал бы обычный маркет-мейкер, но с ценами, рассчитанными по алгоритму постоянного произведения.

В ходе моделирования предполагается, что AMM-агент всегда остается в рынке и выполняет сделки согласно описанной стратегии.
Исходный объем его ликвидности (запасы $x$, $y$) влияет на глубину рынка: в теории, чем больше резервов у AMM, тем меньшее проскальзывание наблюдается при крупных сделках, и тем более стабилен рынок.
Для простоты комиссии за обмен приняты равными нулю, так же как и у классического маркет-мейкера, чтобы сфокусироваться именно на различиях алгоритмов маркет-мейкинга.

Чтобы исключить нереалистичные сценарии мгновенного выбрасывания всей ликвидности и зацикливания вычислений, введены ограничения, отражающие риск-менеджмент поставщика ликвидности: за один шаг агент может использовать не более фиксированной доли запасов в рыночных операциях, а сетка лимитных ордеров формируется только на ограниченную долю инвентаря и обрезается по числу заявок.
Таким образом, глубина, предлагаемая AMM, пропорциональна текущим резервам, а количество вычислений на шаг остаётся конечным и сопоставимым с поведением реального маркет-мейкера, ограниченного риск-лимитами на объём и частоту котировок.

\subsection*{Механизм ценообразования AMM в HFMM}

В дополнение к пулу с постоянным произведением реализован гибридный механизм HFMM (Curve CryptoInvariant), который сочетает свойства инвариантов stableswap и constant-product и обеспечивает более концентрированную ликвидность вокруг точки равновесия \cite{egorov2021cryptoswap}.
Пусть текущие запасы актива $A$ равны $x$, актива $B$ — $y$, число активов $N=2$.
Инвариант задаётся уравнением:
\[
F(x,y,D) = K(x,y,D)\,D\,(x+y) + x\,y - K(x,y,D)\,D^2 - \left(\tfrac{D}{2}\right)^2 = 0,
\]

где $D$ — параметр кривой, а

\[
K(x,y,D) = A \cdot K_0 \cdot \frac{\gamma^2}{(\gamma + 1 - K_0)^2}, \qquad K_0 = \frac{x\,y\,2^2}{D^2},
\]
%Последнюю строчку уточнить;
$A$ — коэффициент амплификации, $\gamma$ — малый параметр формы.
Значения $A$ и $\gamma$ подбираются из безопасного диапазона, рекомендованного в whitepaper, чтобы избежать некорректной кривизны.
Цена определяется как маргинальное отношение $\frac{dy}{dx}$ вдоль решения $F = 0$: при изъятии $\Delta x$ из пула новое $y'$ находится решением $F(x + \Delta x, y', D) = 0$, и цена сделки равна $\frac{y-y'}{\Delta x}$.

Чтобы обеспечить устойчивость вычислений, решение по $D$ и по $\frac{x}{y}$ проводится численным бинарным поиском по знаку $F$ без производных (вместо метода Ньютона в оригинальном методе).
Это снижает риск расходимости при больших дисбалансах и делает вычисления ещё более простыми.
Комиссия в моделировании равна нулю, чтобы сравнение с классическим маркет-мейкером и Uniswap AMM было корректным.

\subsection*{Реализация AMM-стратегии через лимитные и рыночные ордера в HFMM}

HFMM-агент использует ту же схему взаимодействия с книгой заявок, что и Uniswap-агент, но с поправкой на отличия алгоритмов ценнобразования.
Таким образом получаем алгоритм:
\begin{itemize}
    \item Рыночные ордера: если в книге есть рыночные заявки, выгодные относительно текущей цены пула (рассчитанной вдоль $F = 0$), агент исполняет их в объёме, не превышающем заданную долю резервов.
    Котировка определяется разницей запасов до/после решения $F(x+\Delta x, y', D) = 0$ (или симметрично по $x'$ при вносе $B$), численно найденного бинарным поиском.
    Это даёт согласованную цену сделки с кривой HFMM.
    \item Лимитные ордера: агент строит дискретную сетку цен вдоль кривой $F = 0$, итеративно шагая по единичному объёму и пересчитывая $y'$ (или $x'$) через бинарный поиск.
    Полученная последовательность цен размещается в книге как серия мелких ордеров на покупку/продажу, имитируя непрерывную кривую HFMM.
\end{itemize}
Ограничения риск-менеджмента совпадают с Uniswap-агентом -- на шаг ограничена доля резервов для рыночных операций, а сетка лимитных ордеров строится только на части резерва агента в пуле и обрезается по количеству заявок.
За счёт инварианта HFMM ликвидность более концентрирована вокруг равновесной точки, поэтому при равном объёме резервов глубина возле равновесной точки выше, чем у Uniswap AMM.

\section*{Проведение экспериментов}

Модель реализована на языке Python. Исходный код экспериментов доступен в открытом репозитории \cite{ai360project2025}.
Для проверки гипотез проведены серии симуляций в двух режимах: (a) с участием классического маркет-мейкера и (b) с участием AMM-агента (Uniswap или HFMM).
Остальной состав агентов и начальные условия рынка фиксированы одинаково в обоих случаях для корректного сравнения, но они различны и перебираются в различных сериях симуляции.
Длительность каждой симуляции разбита на дискретные шаги (эпохи), в течение которых агенты в случайном порядке переходят к совершению сделок.
Модель поддерживает генерацию внешних шоков – событий, вызывающих резкое изменение рыночной ситуации.

Рассмотрены два типа шоковых сценариев, соответствующие выдвинутым гипотезам:

\begin{itemize}
    \item Ценовой шок.
    В определенный момент времени в модель вводится скачкообразное изменение фундаментальной ценности актива (например, негативная новость снижает справедливую цену) на $1\%-90\%$ (сначала с шагом 1\%, после 5\% - с шагом в 5\%).
    После шока наблюдается процесс восстановления -- трейдеры адаптируют свои заявки к новой информации, а маркет-мейкер поддерживает ликвидность во время повышенной волатильности.
    Фиксируются показатели:

    (1) Время стабилизации –- количество шагов, через которое цена перестает существенно отклоняться и колеблется вокруг нового равновесия.
    % Относительное? TODO.
    (2) Новый уровень цены –- относительное изменение установившейся цены актива по сравнению с исходным уровнем до шока.

    \item Шок ликвидности (уход агентов).
    Имитация резкого сокращения числа активных участников торгов.
    В момент шока из системы удаляется определенная доля случайно выбранных трейдеров (кроме маркет-мейкера).
    Рассматриваются случаи ухода от 10\% до 90\% агентов равномерно по всем типам.
    Такое событие отражает, например, уход с рынка ряда крупных игроков или падение интереса к торгам.
    Уменьшение числа контрагентов вызывает снижение ликвидности и может приводить к резким колебаниям цены из-за дисбаланса спроса и предложения.
    После также шока измеряются время стабилизации и новый уровень цены.
\end{itemize}

Каждый эксперимент повторялся на 10000 прогонов со случайными сидами для усреднения результатов.

\section*{Детектирование паники и окончание шока}

Для количественного сопоставления режимов необходимо фиксировать начало и завершение паники.
Мы используем двухэтапный подход, сочетающий идеи контрольных карт Шухарта \cite{shewhart1931economic} и анализа реализованной волатильности \cite{andersen2003modeling}: сперва выявляем резкий рост локальной волатильности относительно базовой, затем определяем момент стабилизации на основе совокупности критериев по тренду, разбросу и нахождению цены в нормативном коридоре.

\subsection*{Детектор паники}

Пусть $P_t$ — цена актива на шаге $t$, а $r_t = \ln(P_t/P_{t-1})$ — логарифмическая доходность.
Рассматриваются две скользящие дисперсии: короткая $\sigma^{(s)}_t$ по окну длиной $W_s$ и базовая $\sigma^{(b)}_t$ по окну $W_b \gg W_s$.
В экспериментах используются $W_s = 10$ (сопоставимо с временем реакции агентов) и $W_b = 60$ (покрывает докризисный период).
Индикатор паники определяется как отношение
\[
R_t = \frac{\sigma^{(s)}_t}{\sigma^{(b)}_t}.
\]
Если $R_t$ превышает порог $K_p$ (в настройках симулятора это $2{,}5$) на протяжении не менее $q$ последовательных шагов ($q=2$), фиксируется начало паники.
Такой критерий эквивалентен контрольной карте с динамической нормировкой, где $\sigma^{(b)}_t$ задаёт нормальный режим, а $\sigma^{(s)}_t$ отслеживает мгновенные всплески.
Отношение волатильностей делает детектор инвариантным к масштабу и позволяет корректно сопоставлять режимы с классическим маркет-мейкером и с AMM, где кривая ликвидности меняет чувствительность цены к объёму.

\subsection*{Определение окончания шока}

Пусть $t_0$ — момент шока. Для каждого кандидата $t > t_0$ считаются:


\textbf{Докризисная база:} по промежутку $[t_0 - W_{\text{ref}}, t_0)$ вычисляются средняя цена $p_{\text{ref}}$ и стандартное отклонение $\sigma_{\text{ref}}$, задающие ширину допустимого коридора.
\textbf{Нормативная траектория:} если доступна фундаментальная цена, она используется напрямую, а иначе строится экспоненциальное скользящее среднее $p_{\text{norm}}(t)$ с коэффициентом $\alpha$, обновляемое только по уже наблюдавшимся данным.


Далее проверяются следующие условия:
\begin{enumerate}
    \item Линейная регрессия $P_\tau = a + b\tau$ даёт наклон $|b| \le \theta$ или статистически незначимый $b$ (определяем отсутствие тренда).
    \item Стандартное отклонение окна $\hat{\sigma}_{\text{tail}} \le v \cdot \sigma_{\text{ref}}$ (определяем возврат волатильности к базовому уровню).
    \item Цена $P_t$ лежит в полосе $[p_{\text{norm}}(t) \pm \max(\kappa\sigma_{\text{ref}}, \varepsilon|p_{\text{ref}}|)]$, исключая преждевременную фиксацию.
\end{enumerate}


Если все условия выполняются $c$ раз подряд, момент последнего окна считается окончанием шока $t_{\text{end}}$, а устойчивый уровень $p^\ast$ вычисляется как среднее цены в окне подтверждения.
Такой комбинированный алгоритм учитывает и уровень, и разброс, и тренд, что особенно важно в агентно-ориентированной модели, ведь отдельные типы трейдеров могут создавать временные аномалии, но только длительное соблюдение всех критериев отражает реальное восстановление рынков как с AMM, так и с классическим маркет-мейкером.

\subsection*{Измерение качества ликвидности}

Произведено 10000 опытов, призванных показать различия в качестве ликвидности классического маркет-мейкера и двух реализаций AMM. В каждом опыте были измерены:

\begin{enumerate}
    \item Средний и относительный спред ($ask - bid$ и $\frac{ask-bid}{mid}$), оценка на их стандартные отклонения.
    \item Средний объём на лучших котировках.
    \item Глубина в узком ценовом коридоре вокруг мида ($\pm 1\%$).
    \item Среднее число заявок в книге (bid/ask).
    \item Отношение спреда к суммарному объёму на лучших котировках (отражает, насколько дорогой становится предельная ликвидность).

\end{enumerate}

%Так что в итоге. TODO
Статистическая значимость проверяется парными тестами на равенство средних (t-тест или Wilcoxon при ненормальности) и тестом Левена на равенство дисперсий.
Такая постановка позволяет оценить, выигрывает ли AMM по качеству предоставляемой ликвидности даже без воздействия шоков.

\section*{Результаты и обсуждение}

\section*{Тестирование модели}

\info{TODO}